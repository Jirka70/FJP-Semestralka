\documentclass[11pt]{beamer}
\usetheme{Berlin}
\usepackage[utf8]{inputenc}
\usepackage{amsmath}
\usepackage{amsfonts}
\usepackage{amssymb}
\usepackage{svg}
\usepackage{graphicx}
\author{Jan Vašátko, Jiří Třesohlavý}


% Balíček listings pro zobrazení zdrojového kódu
\usepackage{listings}
\usepackage{xcolor}

% Nastavení syntaxe pro zvýraznění "latinských" klíčových slov
\lstdefinelanguage{LatinJava}{
    keywords={classis, publicus, staticus, vacuum, principalis, si, aliter, pro, int},
    keywordstyle=\color{blue}\bfseries,
    commentstyle=\color{gray}\itshape,
    stringstyle=\color{red},
    morecomment=[l]{//},      % řádkové komentáře
    morecomment=[s]{/*}{*/},  % blokové komentáře
    morestring=[b]",          % řetězce jsou mezi uvozovkami
}

% Nastavení vzhledu pro kód
\lstset{
    language=LatinJava,
    basicstyle=\ttfamily\tiny, % základní styl kódu
    numbers=left,               % čísla řádků vlevo
    numberstyle=\tiny\color{gray},
    stepnumber=1,
    tabsize=2,
    breaklines=true,
    showstringspaces=false,
    frame=single,
}




\title{Překladač latinské Javy do instrukcí PL/0}
\setbeamercovered{transparent} 
\setbeamertemplate{navigation symbols}{} 
\institute{Fakulta aplikovaných věd} 


\begin{document}

\begin{frame}
\titlepage
\end{frame}

%\begin{frame}
%\tableofcontents
%\end{frame}

\begin{frame}[fragile]
\frametitle{Příklad Javy v latině}
\begin{itemize}
	\item Přepsání klíčových slov z angličtiny do latiny
\end{itemize}

\begin{lstlisting}
classis ClassisSalve {
    publicus staticus vacuum principalis(String[] argumenta) {
        System.out.println("Salve, mundi!");
        
        int numerus = 10;
        
        si (numerus > 5) {
            System.out.println("Numerus est maior quam quinque.");
        } aliter {
            System.out.println("Numerus non est maior quam quinque.");
        }
        
        pro (int i = 0; i < numerus; i++) {
            System.out.println("Iteratio: " + i);
        }
    }
}
    \end{lstlisting}
\end{frame}

\begin{frame}{Práce s celými čísly}
\begin{itemize}
	\item Budeme volit znaménkový \texttt{int} (32-bitový)
	\item Zvažujeme i práci s jinými typy (\texttt{long, short, byte})
	\item Číselné literály mohou být definovány v násl. soustavách:
	\begin{itemize}
		\item Desítková -- \texttt{42}
		\item Šestnáctková -- \texttt{0xFF00FF}
		\item Dvojková -- \texttt{0B101}
		\item Osmičková -- \texttt{0377}
	\end{itemize}
	\item Nevyhýbáme se ani římským číslicím 
\end{itemize}

\end{frame}
\begin{frame}{Práce s reál. čísly}
\begin{itemize}
	\item Pro reálná čísla využijeme typ \texttt{float} -- 32-bitový single-precision formát IEEE 754
	\item Literály pro reálná čísla se zapisují:
	\begin{itemize}
		\item S desetinou tečkou -- \texttt{3.14159}
		\item S exponentem -- \texttt{3.14e6}
	\end{itemize}	
	\item  Pokud je reál. číslo ve výrazu s celým číslem:
	\begin{itemize}
		\item Celé číslo se implicitně přetypuje na reál. číslo
		\item Následně se provádí aritmetika s reál. číslem
	\end{itemize}
\end{itemize}
\end{frame}
\begin{frame}{Reprezentace logic. typů}
\begin{itemize}
	\item Pro logické hodnoty použijeme datový typ \texttt{boolean}
	\item Tento datový typ může nabývat jen hodnot \texttt{true} nebo \texttt{false}
\end{itemize}
\end{frame}
\begin{frame}{Reprezentace znaků}
\begin{itemize}
	\item Pro znakové typy použijeme dat. typ \texttt{char} -- 16-bitový
	\item Literály se zapisují jako
	\begin{itemize}
		\item Jednotlivé znaky v uvozovkách (\texttt{'a'})
		\item Unicode sekvence 
	\end{itemize}
	\item Ve výrazech s typem \texttt{int} bude implicitně přetypován na \texttt{int}
\end{itemize}
\end{frame}
\begin{frame}{Reprezentace polí}
\begin{itemize}
        \item Pole budeme vytvářet stejně jako v Javě:
        \begin{itemize}
            \item \texttt{typ[] p = new typ[velikost];} pro vytvoření pole určité velikosti.
            \item \texttt{typ[] p = \{výčet prvků\};} pro inicializaci pole při deklaraci.
        \end{itemize}
        \item Pole budou implementována pomocí referenční proměnné, tedy stejně jako v Javě.
        \item Přiřazování mezi poli probíhá přiřazením referencí, ne kopírováním obsahu.
        \item Implementujeme operátory:
        \begin{itemize}
            \item \texttt{[]} pro přístup k prvkům pole.
            \item \texttt{.length} pro zjištění délky pole.
        \end{itemize}
    \end{itemize}
\end{frame}
\begin{frame}{Reprezentace řetězců}
\begin{itemize}
        \item Řetězce můžeme implementovat jako \texttt{Filum r = "retezec";}.
        \item V paměti budou uloženy jako \texttt{null-terminated} pole znaků.
        \item K dispozici budou následující operátory:
        \begin{itemize}
            \item \texttt{[]} pro přístup k jednotlivým znakům,
            \item \texttt{.longitudo} pro zjištění délky řetězce,
            \item \texttt{==} pro porovnání řetězců.
        \end{itemize}
    \end{itemize}
\end{frame}
\begin{frame}{Operátory a řídící struktury}
\begin{itemize}
        \item \textbf{Aritmetické operátory:}
        \begin{itemize}
            \item \texttt{+}, \texttt{-} (včetně unárního), \texttt{*}, \texttt{/}, \texttt{\%}
            \item Inkrementace a dekrementace: \texttt{++}, \texttt{--}
            \item Zkrácené přiřazení: \texttt{+=}, \texttt{-=}, \texttt{*=}, \texttt{/=}, \texttt{\%=}
        \end{itemize}
        \item \textbf{Relační operátory:}
        \begin{itemize}
            \item \texttt{<}, \texttt{<=}, \texttt{==}, \texttt{>}, \texttt{>=}
        \end{itemize}
        \item \textbf{Logické operátory:}
        \begin{itemize}
            \item \texttt{\&}, \texttt{\&\&}, \texttt{|}, \texttt{||}
        \end{itemize}
        \item \textbf{Řídicí struktury:}
        \begin{itemize}
            \item \texttt{if}, \texttt{if-else}, \texttt{switch}, ternární operátor \texttt{? :}
            \item Smyčky: \texttt{for}, \texttt{while}
        \end{itemize}
    \end{itemize}
\end{frame}
\begin{frame}[fragile]
\frametitle{Gramatika našeho jazyka}
\begin{itemize}
	\item Pravidla pro gramatiku našeho jazyka budeme brát z AntLR z lexeru a parseru
	\item Téměř totožná s pravidly klasické Javy
\end{itemize}
\vspace{.5cm}
Definování identifikátoru a výrazu:
\begin{lstlisting}
Identifier
    : [a-zA-Z_] [a-zA-Z_0-9]*
    ;
\end{lstlisting}
\begin{lstlisting}
expression
    : primaryExpression // napr.: promena, cislo
    | expression operator expression 
    | '(' expression ')'    
    ;
\end{lstlisting}
\end{frame}
\begin{frame}[fragile]
\frametitle{Pravidla pro cyklus \texttt{for} a \texttt{if} statement}
\begin{lstlisting}
forStatement
    : 'pro' '(' forControl ')' statement
    ;

forControl
    : forInit? ';' expression? ';' forUpdate?
    ;

forInit
    : variableDeclaration
    | expressionList
    ;

forUpdate 
    : expressionList
    ;
\end{lstlisting}
\begin{lstlisting}
ifStatement
    : 'num' (' expression ')' statement ('else' statement)?
    ;
\end{lstlisting}
\end{frame}

\begin{frame}[fragile]
\frametitle{Deklarace metod}
\begin{lstlisting}
methodDeclaration
    : type Identifier '(' parameterList? ')' block
    ;

parameterList
    : parameter (',' parameter)*
    ;

parameter
    : type Identifier
    ;

block
    : '{' blockStatement* '}'
    ;

blockStatement
    : variableDeclaration
    | statement
    ;
statement
    : ifStatement
    | forStatement
    | expressionStatement
    | returnStatement
    ;
\end{lstlisting}
\end{frame}
\begin{frame}{Zdroje (APA citation)}
\begin{itemize}
	\item \texttt{Lexer}: Antlr. (n.d.). grammars-v4/java/java9/Java9Lexer.g4 at master · antlr/grammars-v4. GitHub. https://github.com/antlr/grammars-v4/blob/master/java/java9/Java9Lexer.g4
	\item \texttt{Parser}:
	Antlr. (n.d.). grammars-v4/java/java9/Java9Lexer.g4 at master · antlr/grammars-v4. GitHub. https://github.com/antlr/grammars-v4/blob/master/java/java9/Java9Parser.g4
\end{itemize}
\end{frame}
\end{document}